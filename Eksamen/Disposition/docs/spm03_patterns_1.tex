\section{Patterns 1 - GoF Strategy + GoF Template Method}

\subsection{Fokuspunkter}

\begin{itemize}
	\item Redegør for, hvad et Software Design Pattern er.
	\item Sammenlign de to design patterns GoF Strategy og GoF Template Method - hvornår vil du anvende hvilket, og hvorfor?
	\item Vis et designeksempel på anvendelsen af GoF Strategy.
	\item Redegør for, hvordan anvendelsen af GoF Templete fremmer godt SW design.
	\item Redegør for, hvilke(t) SOLID-princip(per) du mener anvendelsen af GoF Strategy understøtter.
\end{itemize}

\subsection{Hvad er et Software pattern?}

\derp

\subsection{Sammenlign de to design patterns GoF Strategy og GoF Template Method - hvornår vil du anvende hvilket, og hvorfor?}

\subsection{Vis et designeksempel på anvendelsen af GoF Strategy}

\subsection{Redegør for, hvordan anvendelsen af GoF Templete fremmer godt SW design}
Hvis man har et system bestående af nogle klasser, hvor disse klasse funktionalitet kun afviger lidt fra hinanden. Så kan \textit{Template pattern} bruges. Via dette pattern kan et fast ''programflow'' defineres. Når dette ''flow'' så er fastlagt skal klasserne bare implementere/ændre (nok via override) de metoder som de ikke er tilfredse med.

\subsubsection{Eksempel på Template pattern}
Herunder er en abstrakt klasse der har metoder, som de mange spil bruger/følger. Eksemplet er taget fra \href{https://en.wikipedia.org/wiki/Template_method_pattern#Example_in_Java}{wikipedia} om template pattern

\begin{lstlisting}
abstract class Game {
	// Hook methods. Concrete implementation may differ in each subclass
	protected int PlayerCount;
	abstract void InitializeGame();
	abstract void MakePlay(int player);
	abstract void EndOfGame();
	abstract void AnnounceWinner();
	
	// A template method:
	public final void PlayGame(int playerCount)	{
		PlayerCount = playerCount;
		InitializeGame();
		int j = 0;
		while(!EndOfGame()) {
			MakePlay(j);
			j = (j + 1) % PlayerCount;
		}
		PrintWinner();
	}
}
\end{lstlisting}

Herunder ses så hvordan den specifikke implementering af den abstrakts klasse kan laves.

\begin{lstlisting}
class Monopoly : Game {
	/* Implementation of necessary concrete methods */
	void initializeGame() {
		// Initialize players
		// Initialize money
	}
	void makePlay(int player) {
		// Process one turn of player
	}
	boolean endOfGame() {
		// Return true if game is over 
		// according to Monopoly rules
	}
	void printWinner() {
		// Display who won
	}
	/* Specific declarations for the Monopoly game. */
	// ...
}
\end{lstlisting}

\subsection{Redegør for, hvilke(t) SOLID-princip(per) du mener anvendelsen af GoF Strategy understøtter}








