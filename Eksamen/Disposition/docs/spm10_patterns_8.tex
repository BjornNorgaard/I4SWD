\section{Patterns 6 - Redegør for følgende concurrency mønstre}

\subsection{Fokuspunkter}

\begin{itemize}
	\item Parallel Aggregation.
	\item MapReduce.
\end{itemize}

\subsection{Parallel Aggregation}

\subsection{MapReduce}
MapReduce er en måde at arbejde på meget store datasæt. Der arbejdes med MapReduce parallelt på dataen.

\subsubsection{4 steps}
Vi kan opdele MapReduce modellen i 4 steps:

Eksempel med spillekort:

Vi ønsker at vide hvor mange af hver kulør der findes i vores stak.
\begin{enumerate}
	\item \textbf{Distribuer source data til forskellige nodes.}\\
	Fordel stakken af kort ud på alle noder.
	\item \textbf{Map data - Repræsenter data i key-value par.}\\
	Hver node tæller hvor mange af hver kulør den har fået tildelt. Dette repræsenteres med key-value par. eg. Kulør (key) og kortværdi (value).
	\item \textbf{Gruppér data.} \\
	Grouperen har til ansvar at gruppere mappernes key-value par. Her tages alle klør par og sættes i samme gruppe, alle ruder par sættes sammen osv.
	\item \textbf{Reducer eller merge  data fra grouperen.}\\
	Her tælles sammen hvor mange af hver kulør der er i de grupperede data.
\end{enumerate}
